\documentclass[a4paper,10pt,twoside]{article}

\title{\huge \textbf{FPGA notes}}
\author{Nicola Zaupa}
\date{\today}

%=============================================================
% ---PACKAGES --- %
%=============================================================

% -- BASIC -- %
\usepackage[utf8]{inputenc} % input encoding
\usepackage{amsmath,amssymb} % mathematics stuff
\usepackage{amsthm}         % introduce \theoremstyle{}
\usepackage{enumerate}      % enable enumerate environment
% \usepackage{enumitem}       % advance creation of list
\usepackage{cancel}         % \cancel draw a diagonal line
\usepackage{cases}          % introduce \numcases, useful for piecewise stuff
\usepackage{geometry}		% geometry of the page
\geometry{a4paper,top=2cm,bottom=2cm,left=2cm,right=2cm,heightrounded}
\usepackage{hyperref}
\hypersetup{
    colorlinks,
    linkcolor={black},
    citecolor={black},
    urlcolor={black}
}



% -- FLOAT -- %
\usepackage{float}          % improve float objects usage
% \usepackage{subfig}         % take care, may have incompatibility
% \usepackage{subfigure}      % take care part 2
\usepackage{booktabs}       % enhance table layout \toprule
\usepackage{multirow}
\usepackage{multicol}
\usepackage{placeins}       % introduce \FloatBarrier


\usepackage{graphicx}       % introduce utility to modify text (rotating, stretching, ...)
\usepackage[dvipsnames]{xcolor} % improve color capability
\usepackage{rotating}       % allows rotation of any kind
\usepackage{pdfpages}       % easily introduces pdf pages \includepdf
\usepackage{import}         % allow \input of files

\usepackage{verbatim}       % useful when need to insert code
\usepackage{listings}

\usepackage{nicefrac}       % command \nicefrac{}{}
\usepackage{theorem}

\usepackage{cite}           % improve use of \cite



% -- OTHER -- %
% \usepackage{soul}           % introduce \st{overstriking}



%=============================================================
% MATH OPERATORS
%=============================================================

\DeclareMathOperator{\dom}{dom}
\DeclareMathOperator{\sign}{sign}
\DeclareMathOperator{\Sign}{Sign}
\DeclareMathOperator{\argmin}{argmin}
\DeclareMathOperator{\inte}{int}
\DeclareMathOperator{\gph}{gph}
\DeclareMathOperator{\rge}{rge}
\DeclareMathOperator{\e}{e}

%=============================================================
% NEW COMMAND
%=============================================================

% -- MATH -- %

\newcommand{\real}{\ensuremath{\mathbb R}}
\newcommand{\nat}{\ensuremath{\mathbb{N}}}
\newcommand{\integer}{\ensuremath{\mathbb Z}}
\newcommand{\integers}{\ensuremath{\mathbb Z}}
\newcommand{\0}{\textbf{0}}
\newcommand{\1}{\textbf{1}}
\newcommand{\smallmat}[1]{\ensuremath{\left[\begin{smallmatrix}#1\end{smallmatrix}\right]}}
\newcommand{\bigmat}[1]{\ensuremath{\begin{bmatrix}#1\end{bmatrix}}}

\newcommand{\A}{\ensuremath{\mathcal{A}}}
\newcommand{\C}{\ensuremath{\mathcal{C}}}
\newcommand{\D}{\ensuremath{\mathcal{D}}}
\newcommand{\F}{\ensuremath{\mathcal{F}}}
\newcommand{\G}{\ensuremath{\mathcal{G}}}
\newcommand{\K}{\ensuremath{\mathcal{K}}}

\newcommand{\coord}{\ensuremath{(z_1,z_2)}}

\def\myminus{\scalebox{0.5}[1.0]{\( - \)}}
\def\myplus{\scalebox{1}[1]{\(+\)}}


% -- OTHER -- %

\newcommand{\overbar}[1]{\mkern 1.5mu\overline{\mkern-1.5mu#1\mkern-1.5mu}\mkern 1.5mu}
\newcommand{\overbbar}[1]{\bar{\bar{#1}}}
\newcommand\numberthis{\addtocounter{equation}{1}\tag{\theequation}}


% -- ENVIRONMENTS -- %

\newenvironment{example}{\begin{nnexample} \rm }{\hfill \hspace*{1pt}\hfill $\star$\end{nnexample}}
\newenvironment{remark}{\begin{nnremark} \rm }{\hfill \hspace*{1pt}\hfill $\circ$\end{nnremark}}
\newenvironment{proofof}{{\em Proof of }}{\hfill \hspace*{1pt}\hfill $\blacksquare$}

% -- COLORS -- %

\newcommand{\red}[1]{{\color{red}#1}}
\newcommand{\blue}[1]{{\color{blue}#1}}
\newcommand{\green}[1]{{\color{Green}#1}} 
\newcommand{\new}[1]{{\color{Purple}#1}}
\newcommand{\del}[1]{{\color{Gray}#1}}

% \renewcommand{\{}{\textbraceleft}
% \renewcommand{\}}{\textbraceright}

% -- CODE -- %

\newcommand{\code}[1]{{\color{Mulberry}\textbf{\texttt{#1}}}}

\definecolor{Verilog}{RGB}{240,240,240}    
\lstdefinestyle{verilog-style}
{
    language=Verilog,
    basicstyle=\small\ttfamily,
    keywordstyle=\color{RubineRed}\bfseries,
    backgroundcolor = \color{Verilog},
    frame = none,
    gobble = 14
}




\begin{document}
    
    \maketitle
    \tableofcontents

    \vfill
    \noindent
    Info about the setup:
    \begin{enumerate}[a.]
        \item BOARD: DE4 Stratix IV Development BOARD
        \item CORE: Stratix IV GX
        \item CHIP: EP4SGX230KF40C2
        \item LANGUAGE: Verilog
    \end{enumerate}

    \newpage



\section{Installing}
    Some notes about installing the necessary softwares.\\


    \noindent\textbf{LICENSE}
    \begin{enumerate}
        \item go to \url{https://licensing.intel.com/psg/s/} and access with the credential already linked to the device.
        \item go to \textit{Computers and License Files} to create a new license (licenses are linked to the PC and also to the ethernet network (MAC address)
        \item add a new Computer (\textit{New})
            \begin{itemize}
                \item Computer name: right click on \code{start} and then \textit{System} $\rightarrow$ \textit{Device name}
                \item Computer Type: NIC ID, go to Command Line in Win ``\code{cmd}'' and type \code{ipconfig/all}. Find the \code{Physical Adress} of the interface (connection) that you are using (you also find the host name) [probably other interfaces on the same PC can be added]
            \end{itemize}
        \item go to \textit{Licenses}-\textit{All Licenses} and choose the one related to the product you need to activate
        \item \code{Generate License} $\rightarrow$ \textit{Assign an Existing Computer} [you should be able to rehost\dots otherwise you're \dots] 
        \item Choose the computer 
        \item 
    \end{enumerate}


\section{Coding}
    I code in Verilog. Now testing the coding in Visual Studio Code.

    Type of signal:
    \begin{itemize}
        \item \code{wire} can used with \code{assign}: \code{assign a = b \& c}
        \item \code{reg}: values can be assigned with procedural blocks (\code{always, initial} : $<=$) but not with continuous (\code{assign}). Because \code{reg} is capable of storing and does not require to be driven continuously.
    \end{itemize}

    \noindent 
    Initialize a variable
    \begin{itemize}
        \item at declaration: \code{reg [31:0] data = 32'hdead\_cafe;}
        \item in \code{initial begin}: \code{data = 32'hdead\_cafe;}
        \item can also initialize \code{wire}
    \end{itemize}
    Avoid to do it both, since is not sure which is executed first. Underscores `\code{\_}' are not counted as bytes, their are useful to separate long numbers.

    Could be interesting \code{assign-deassign} and \code{force-release}.

    \vspace{30pt}

    Structure for the loop \code{always}

    

    \begin{center}
        \textbf{Operators}\\[10pt]
        \begin{tabular}{ccl|cl|cl}
            \code{\&\&}   & \code{\&}       & and  & \code{*}   & multi    & \code{===}  & equal, including x and z \\
            \code{||}     & \code{|}        & or   & \code{/}   & divided  & \code{!==}  & not equal, including x and z  \\
            \code{!}      & \code{$\sim$}   & not  & \code{\%}  & modulo   & \code{==}   & equal, can be unknown \\
                          & \code{$\wedge$} & xor  & \code{**}  & power    & \code{!=}   & not equal, can be unknown  
        \end{tabular}
    \end{center}
    \begin{center}
        \textbf{Shift}\\[10pt]
        \begin{tabular}{lccl}
             Logical shift      & \code{<<}   &  \code{>>}   &  add zeros \\
             Arithmetic shift   & \code{<<<}  &  \code{>>>}  &  keep the (sign?) bit (not sure about \code{<<<})
        \end{tabular}\\[10pt]
        \begin{tabular}{lcl}
            \code{data = 00101} & $\rightarrow$ & \code{data << 1 \ = 01010} \\
                                &               & \code{data << 2 \ = 10100} \\
                                &               & \code{data <<< 1 = 01011}  \\
       \end{tabular}
    \end{center}
    
    \textbf{Conditional operator}: \code{assign out = cond ? true : false}

    \textbf{Concatenation}: use the curly brackets 
    \code{ $\{ ~ \}$ } 
    $\ \rightarrow\ $ 
    \code{$\{a, b, c\left[1:0\right], 2'b00, \{2\{a\}\}\}$}

    \textbf{Replication}: \code{$\{y,y,y\}$ = $\{3\{y\}\}$}, it can be nested.

\section{Simulation}
    Describe how to setup a simulation in Modelsim (Questasim)

    $\rightarrow$ is possible to generate a file ``\textit{*.do}'' to lunch for the simulation (should simplify the steps).


\section{Quartus}
    \begin{itemize}
        \item go to the RTL\footnote{Register Transfer Level} view
        \item assign the PIN (use csv file to define the connection)
    \end{itemize}

    \textbf{NEW PROJECT}\\
    \begin{enumerate}
        \item File \code{->} New Project Wizard
        \item Select/Create the folder for the project
        \item Give a name to the project and select/create the TOP file of the code
        \item Select \code{Empty Project}
        \item In case select existing files (blocks), then \code{Next}
        \item Select the device on the board
        \item If necessary link simulation programs or others
        \item Finish 
    \end{enumerate}

    \textbf{PIN ASSIGNEMENT}
    \begin{enumerate}
        \item \textit{Assignments}
        \item \textit{Import Assignment}
        \item Select the \code{*.csv} file, then \textit{Ok}
    \end{enumerate}

    Press the button play to compile the code

    \textbf{LOAD the CODE}
    \begin{enumerate}
        \item Open the Programmer [small image]
        \item Press on \code{Hardware Setup}
        \item On \textit{Currently selected hardware} select \code{USB-Blaster [USB-0]}. If not present maybe is necessary to update the driver of the USB blaster (on your PC go to device-Select the USB-Update Driver-Make windows search in the installation folder of Quartus)
        \item Press \code{Start} to load the code [A blue led should be ON on the board]
        \item 
    \end{enumerate}


   \textbf{POSSIBLE PROBLEMS}\\
   Error: ``\code{Error: Can't open project -- you do not have permission to write to all the files or create new files in the project's database directory}'' $\rightarrow$ look at \url{https://www.intel.com/content/www/us/en/support/programmable/articles/000084612.html}


\section{Data flow and manipulation}
    $i_C$ and $v_C$ are measured through the ADC. Chain of amplification:
        \begin{enumerate}
            \item analog electronics (attenuation) to match ADC requirements
            \item ADC (from analog to digital)
        \end{enumerate}

\section{Program blocks}

    \subsection{TOP block}
        Is the block that is directly interfaced with the hardware. It defines the connection with the output PINs, and manage the main code. Then the code can take advantages of blocks in order to simplify it by making it more clearer and simplifying the debugging.

    \subsection{debounce}
        
        \begin{minipage}{0.1\textwidth}
            $ $
        \end{minipage}
        \begin{minipage}{0.3\textwidth}
            \begin{lstlisting}[style={verilog-style}]
                module debounce (
                    output o_switch,
                    input i_clk,
                    input i_reset,
                    input i_switch
                );
            \end{lstlisting}
        \end{minipage}\quad
        \begin{minipage}{0.5\textwidth}
            Given an single bit input signal, the output switch to the input signal only when the input is stable for a certain amount of time. The time is defined trough a counter, hence depends on the counter limit and on the clock that activates the counter.
        \end{minipage}

    % \code{
    % module debounce (
    %     output o_switch,
    %     input i_clk,
    %     input i_reset,
    %     input i_switch
    % );
    % }

    % \begin{verbatim}
    % module debounce (
    %     output o_switch,
    %     input i_clk,
    %     input i_reset,
    %     input i_switch
    % );
    % \end{verbatim}


    \subsection{ABS}
        Compute the absolute value of a number.

    \subsection{dead\_time}
        Introduce a delay on a rising edge of a signal.

    \subsection{hybrid\_control}
        Implement the control to sustain the self oscillation.

        \begin{itemize}
            \item hybrid\_control\_theta $-$ frequency modulation 
            \item hybrid\_control\_phi  $-$ amplitude modulation
            \item hybrid\_control\_theta\_phi $-$ combination of amplitude and frequency modulation
        \end{itemize}

    \subsection{LPF}
        Implement a digital Low Pass Filter on a signal. Important parameter is the number of samples that are considered. It is a moving average filter.
% \begin{code}
% module LPF (
%     o_mean,      // [32bit-signed] mean on 4 samples
%     i_clock,     // for sequential behavior
%     i_RESET,     // reset signal
%     i_data,      // [32bit-signed] input data to be filtered
%     );
% \end{code}

    \subsection{peak\_detector}
        Detects a peak on a signal by computing the two points derivative.

    \subsection{PI}
        Block that implement the PI controller. 
        Take the error as input and give the correction as output.

    \subsection{trigonometry}
        Compute sine and cosine of an angle in degree(?)


    \subsection{PLL}
        Manage the clock generation in the FPGA. Script generated by Quartus.

    \subsection{angle\_control}
        Need to be designed, now there are two separate version, one for $\theta$ and one for $\varphi$.

        The idea is to create a single block that has as inputs
        \begin{itemize}
            \item button signal
            \item upper limit
            \item lower limit
            \item step magnitude
            \item reset button
        \end{itemize}




\section{PIN mapping of the board}
    Maybe some figures on the PIN mapping that could be useful to have at hand.



\end{document}


% \begin{minipage}{0.5\textwidth}
%     $\delta$
% \end{minipage}

% \lstdefinestyle{verilog-style}
% {
%     language=Verilog,
%     basicstyle=\small\ttfamily,
%     keywordstyle=\color{blue}\bfseries,
%     % identifierstyle=\color{black},
%     % commentstyle=\color{Green},
%     backgroundcolor = \color{Verilog},
%     % numbers=left,
%     % inputpath=code/,
%     % numberstyle=\scriptsize\color{black},
%     % numbersep=1pt,
%     % tabsize=1,
% %		moredelim=*[s][\colorIndex]{[}{]},
%     % literate=*{:}{:}1,
%     % breaklines = true,
%     frame = none%TB%single%,
%     % framesep = 4pt,
%     % framerule = 0.4pt%,
% }
% % \newcommand{\code}[1]{
% % \begin{minipage}{0.1\textwidth}
% % $ $
% % \end{minipage}
% % \begin{minipage}{0.3\textwidth}
% % \begin{lstlisting}[style={verilog-style}]
% % #1
% % \end{lstlisting}
% % \end{minipage}
% % }